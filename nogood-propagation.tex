% ----------------------------------------------------------------------
\begin{frame}{Outline of {NogoodPropagation}}
\begin{itemize}
\item Derive deterministic consequences via:
  \begin{itemize}
  \item Unit propagation on $\Delta_P$ and $\nabla$;
  \item Unfounded sets~$U\subseteq\atom{P}$
  \end{itemize}
\item Note that $U$ is \alert{unfounded} if $\EB{U}{P}\subseteq\flits{\ass}$
  \begin{itemize}
  \item \structure{Note} \  For any $a\in U$, we have $(\lambda(a,U)\setminus\{\Tsigned{a}\})\subseteq\ass$
  \end{itemize}
\pause
\item An ``interesting'' unfounded set~$U$ satisfies:
\[
  \emptyset\subset U \subseteq (\atom{P}\setminus\flits{\ass})
\]
\item Wrt a fixpoint of unit propagation,\\
\pause
      such an unfounded set contains some loop of~$P$
  \begin{itemize}
  \item \structure{Note} \  Tight programs do not yield ``interesting'' unfounded sets !
  \end{itemize}
\pause
\item Given an unfounded set $U$ and some $a\in U$,
      adding $\lambda(a,U)$ to $\nabla$ triggers a conflict or
      further derivations by unit propagation
  \begin{itemize}
  \item \structure{Note} \  Add loop nogoods atom by atom to eventually falsify all $a\in U$
  \end{itemize}
\end{itemize}
\end{frame}
% ----------------------------------------------------------------------
\begin{frame}[shrink=15]%
%\SetAlFnt{\tiny}
\begin{algorithm}[H]
\caption{\textsc{NogoodPropagation}\label{algo:unit:propagation}}
\BlankLine
\Input{A normal program $P$, a set $\nabla$ of nogoods, and an % ordered
       assignment \ass.}
\Output{An extended % ordered
        assignment and set of nogoods.}
\BlankLine
$U := \emptyset$\AlgoComm*{unfounded set}
\BlankLine
\Loop{}{%
  \Repeat{$\Sigma=\emptyset$}{%
    \lIf{$\delta\subseteq \ass$ \ForSome $\delta\in\CN{P}\cup\nabla$}{%
      \Return $(\ass,\nabla)$\AlgoComm*{conflict}
    }
    $\Sigma:=\{\delta\in\CN{P}\cup\nabla \mid \delta\setminus\ass=\{\opp{\sigma}\},\sigma\notin\ass\}$\AlgoComm*{unit-resulting nogoods}
    \lIf{$\Sigma\neq\emptyset$}{%
      \Let{$\opp{\sigma}\in\delta\setminus\ass$ \ForSome $\delta\in\Sigma$}{%
        $\dlevel{\sigma}:= \max(\{\dlevel{\rho}\mid\rho\in\delta\setminus\{\opp{\sigma}\}\}\cup\{0\})$\;
        $\ass:= \ass\circ\sigma$
      }
    }
  }
\BlankLine
  \lIf{$\Loops{P}=\emptyset$}{%
    \Return $(\ass,\nabla)$%\AlgoComm*{no unfounded set $\emptyset\subset U \subseteq\atom{P}\setminus\flits{\ass}$}
  }\;
  $U:= U\setminus\flits{\ass}$\;
  \lIf{$U=\emptyset$}{%
    $U:=\UnFoundedSet{$P,\ass$}$
  }\;
  \lIf{$U=\emptyset$}{%
    \Return $(\ass,\nabla)$\AlgoComm*{no unfounded set $\emptyset\subset U \subseteq\atom{P}\setminus\flits{\ass}$}
  }
  % \lLet{$a\in U$}{%
  %   $\nabla:= \nabla\cup\{\lambda(a,U)\}$\AlgoComm*{(temporarily) record loop nogood}
  \Let{$a\in U$}{%
    $\nabla:= \nabla\cup\{\{\Tsigned{a}\}\cup\{\Fsigned{B}\mid B\in\EB{U}{P}\}\}$\hfill\textit{// record loop nogood}%
  }
}
\end{algorithm}

%%% Local Variables:
%%% mode: latex
%%% TeX-master: "../../../main"
%%% End:

\end{frame}
% ----------------------------------------------------------------------
\begin{frame}{Requirements for \UnFoundedSet}
\begin{itemize}
\item<1-> Implementations of \UnFoundedSet\ must guarantee the following for a result $U$
  \begin{enumerate}
  \item $U\subseteq (\atom{P}\setminus\flits{\ass})$
  \item $\EB{U}{P}\subseteq\flits{\ass}$
  \item $U=\emptyset$ iff there is no nonempty unfounded subset of $(\atom{P}\setminus\flits{\ass})$
  \end{enumerate}

\item<2-> Beyond that, there are various alternatives, such as:
  \begin{itemize}
  \item Calculating the greatest unfounded set
  \item Calculating unfounded sets within strongly connected components
        of the positive atom dependency graph of~$P$
        \medskip
  \item Usually, the latter option is implemented in ASP solvers
  \end{itemize}
\end{itemize}
\end{frame}
% ----------------------------------------------------------------------
\begin{frame}{Example: {NogoodPropagation}}
%
\begin{eqnarray*}
P
& = &
\left\{
  \begin{array}{l}
x  \leftarrow  \neg y\\%[-3pt]
y  \leftarrow  \neg x
\end{array}
\
\begin{array}{l}
u  \leftarrow x,y\\%[-3pt]
u  \leftarrow v%\\%[-3pt]
\end{array}
\
\begin{array}{l}
v  \leftarrow x\\%[-3pt]
v  \leftarrow u,y%\\%[-3pt]
\end{array}
\
\begin{array}{l}
w  \leftarrow \neg x,\neg y\\
\mbox{~}
\end{array}
\right\}
\end{eqnarray*}
%
\[
\footnotesize
\begin{array}[b]{|c|ll|l|}
\hline
\mathit{dl}& \sigma_{\!d} & \overline{\sigma} & \delta
\\\hline\hline
1 & \Tsigned{u} & &
\\\hline
2 & \Fsigned{\{\neg x,\neg y\}} & &
\\
  & & \Fsigned{w}     & \{\Tsigned{w},\Fsigned{\{\neg x,\neg y\}}\}=\clno{w}
\\\hline
3 & \Fsigned{\{\neg y\}} & &
\\
  & & \Fsigned{x}     & \{\Tsigned{x},\Fsigned{\{\neg y\}}\}=\clno{x}
\\
  & & \Fsigned{\{x\}} & \{\Tsigned{\{x\}},\Fsigned{x}\}\in\ClNo{\{x\}}
\\
  & & \Fsigned{\{x,y\}} & \{\Tsigned{\{x,y\}},\Fsigned{x}\}\in\ClNo{\{x,y\}}
\\
  & & \Tsigned{\{\neg x\}} & \{\Fsigned{\{\neg x\}},\Fsigned{x}\}=\clno{\{\neg x\}}
\\
  & & \Tsigned{y} & \{\Fsigned{\{\neg y\}},\Fsigned{y}\}=\clno{\{\neg y\}}
\\
%   & & \Fsigned{\{\neg x,\neg y\}} & \{\Tsigned{\{\neg x,\neg y\}},\Tsigned{y}\}\in\ClNo{\{\neg x,\neg y\}}
% \\
  & & \Tsigned{\{v\}} & \{\Tsigned{u},\Fsigned{\{x,y\}},\Fsigned\{v\}\}=\clno{u}
\\
  & & \Tsigned{\{u,y\}} & \{\Fsigned{\{u,y\}},\Tsigned{u},\Tsigned{y}\}=\clno{\{u,y\}}
\\
  & & \Tsigned{v} & \{\Fsigned{v},\Tsigned{\{u,y\}}\}\in\ClNo{v}
\\
  & & & \{\Tsigned{u},\Fsigned{\{x\}},\Fsigned{\{x,y\}}\}=\lambda(u,\{u,v\})
\\\hline
\end{array}
~\raisebox{1mm}{\KO}
\]
\end{frame}
% ----------------------------------------------------------------------
%
%%% Local Variables:
%%% mode: latex
%%% TeX-master: "../../main"
%%% End:
