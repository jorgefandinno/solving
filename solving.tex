%------------------------------------------------------------
\lecture{Conflict-driven ASP Solving}{solving}
%------------------------------------------------------------
\part{Conflict-driven ASP Solving}
%------------------------------------------------------------
\newcommand{\clno}[1]{\ensuremath{\delta(#1)}}
\newcommand{\ClNo}[1]{\ensuremath{\Delta(#1)}}
\newcommand{\nocl}[1]{\ensuremath{\gamma(#1)}}
\newcommand{\NoCl}[1]{\ensuremath{\Gamma(#1)}}
\newcommand{\sigmaAUX}{\rho}%{\sigma}
%\newcommand{\sigmaUIP}{\sigma}%{\sigma_{\scriptscriptstyle\!\mathit{UIP}}}
%------------------------------------------------------------
\section{Motivation}
%------------------------------------------------------------
\begin{frame}{Motivation}
  \begin{itemize}
  \item<1->\structure{Goal} Approach to computing stable models of logic programs,

    based on concepts from
    \begin{itemize}
    \item Constraint Processing (CP) and
    \item Satisfiability Testing (SAT)
    \end{itemize}

  \item<1->\structure{Idea} View inferences in ASP as unit propagation on nogoods

  \item<1->\structure{Benefits} \
    \begin{itemize}
    \item A uniform constraint-based framework for different\\ kinds of
      inferences in ASP
    \item Advanced techniques from the areas of CP and SAT
    \item Highly competitive implementation
    \end{itemize}
  \end{itemize}
\end{frame}
%------------------------------------------------------------
\section{Boolean constraints}
%------------------------------------------------------------
\begin{frame}{Assignments}
  \begin{itemize}
  \item<1-> An \alert{assignment} \ass\ over $\domain{\ass}=\atom{P}\cup\body{P}$
    is a sequence
    \[
    (\sigma_1,\ldots,\sigma_n)
    \]
    of \alert{signed literals}
    $\sigma_i$ of form \alert{$\Tsigned{v}$} or \alert{$\Fsigned{v}$} for
    $v\in\domain{\ass}$ and $1 \leq i \leq n$
  \item<1,7> $\Tsigned{v}$ expresses that $v$ is \emph{true} and $\Fsigned{v}$ that it is \emph{false}
  \item<2,7> The complement, $\overline{\sigma}$, of a literal $\sigma$ is defined as
    $\overline{\Tsigned{v}} = \Fsigned{v}$ and $\overline{\Fsigned{v}} = \Tsigned{v}$
  \item<3,7> $\ass\circ\sigma$ stands for the result of appending $\sigma$ to \ass
  \item<4,7> Given $\ass=(\sigma_1,\ldots,\sigma_{k-1},\sigma_k,\dots,\sigma_n)$,
        we let $\ass[\sigma_k]=(\sigma_1,\ldots,\sigma_{k-1})$
  \item<5,6,7> We sometimes identify an assignment with the set of its literals
  \item<6,7>
    Given this,
    we access \emph{true} and \emph{false} propositions in \ass\ via
    \[
    \tlits{\ass}
    =
    \{v \in \domain{\ass} \mid \Tsigned{v} \in\ass\}
    \text{ and }
    \flits{\ass}
    =
    \{v \in \domain{\ass} \mid \Fsigned{v} \in\ass\}
    \]
  \end{itemize}
\end{frame}
%------------------------------------------------------------
\begin{frame}{Nogoods, solutions, and unit propagation}
  \begin{itemize}
  \item A \alert{nogood} is a set $\{\sigma_1,\ldots,\sigma_n\}$ of
    signed literals,

    expressing a \alert{constraint} violated by any assignment
    \\
    containing $\sigma_1,\ldots,\sigma_n$
  \item<2-> An assignment \ass\ such that
    \(
    \tlits{\ass} \cup \flits{\ass}
    =
    \domain{\ass}
    \)
    and
    \(
    \tlits{\ass} \cap \flits{\ass}
    =
    \emptyset
    \)

    is a \alert{solution} for a set~$\Delta$ of nogoods,
    if
    $\delta\not\subseteq\ass$ for all $\delta\in\Delta$
  \item<3-> For a nogood $\delta$, a literal $\sigma\in\delta$, and an assignment \ass,
    we say that\\
    \alert{$\overline{\sigma}$} is \alert{unit-resulting} for~$\delta$ wrt \ass,
    if
    \begin{enumerate}
    \item $\delta\setminus\ass=\{\sigma\}$
      and
    \item $\overline{\sigma}\not\in\ass$
    \end{enumerate}
  \item<4-> For a set $\Delta$ of nogoods and an assignment~\ass,
    \alert{unit propagation} is the iterated process of extending \ass\ with
    unit-resulting literals until no further literal is unit-resulting for any
    nogood in $\Delta$
  \end{itemize}
\end{frame}
%------------------------------------------------------------
\section{Nogoods from logic programs}
%------------------------------------------------------------
\subsection{Nogoods from program completion}
%------------------------------------------------------------
\begin{frame}[c]{Nogoods from logic programs\\[-1ex]\normalsize via program completion}

The completion of a logic program $P$ can be defined as follows:
%
\begin{eqnarray*}
  &&
  \{
  v_B \leftrightarrow
  a_1          \wedge \dots \wedge a_m      \wedge
  \neg a_{m+1} \wedge \dots \wedge \neg a_n \mid
  \\&&\qquad\quad
  B \in\body{P} \text{ and }
  B = \{a_1,\dots,a_m,\naf{a_{m+1}},\dots,\naf{a_n}\}
  \}
  \\[6pt]
  &\cup&
  \{
  a \leftrightarrow
  v_{B_1} \vee \dots \vee v_{B_k}
  \mid
  \\&&\qquad\quad
  a\in\atom{P} \text{ and }
  \atbody{P}{a} = \{B_1,\dots,B_k\}
  \}\ ,
\end{eqnarray*}
%
where $\atbody{P}{a}=\{\body{r}\mid r\in P\text{ and }\head{r}=a\}$
\end{frame}
%------------------------------------------------------------
\begin{frame}{Nogoods from logic programs\\[-1ex]\normalsize via program completion}

  \begin{itemize}
%  \item<1-> Let $B=\{a_1,\dots,a_m,\naf{a_{m+1}},\dots,\naf{a_n}\}$ be a body
  \item<1-> The (body-oriented) equivalence
    \[
    v_B \leftrightarrow
         a_1     \wedge\dots\wedge      a_m \wedge
    \neg a_{m+1} \wedge\dots\wedge \neg a_n
    \]
    can be decomposed into two implications:
    \medskip
    \begin{enumerate}
    \item<only@2>\normalsize
      \alert{%
      \(
      v_B \to
           a_1     \wedge\dots\wedge      a_m \wedge
      \neg a_{m+1} \wedge\dots\wedge \neg a_n
      \)}

      \smallskip
      is equivalent to the conjunction of
      \[
      \neg v_B \vee  a_1
      ,\ \dots,\
      \neg v_B \vee  a_m
      ,\
      \neg v_B \vee  \neg a_{m+1}
      ,\ \dots,\
      % \text{ and }
      \neg v_B \vee  \neg a_n
      \]
      and induces the set of nogoods
      \alert{\small%
        \[
        \ClNo{B}
        =
        \{\,\{\Tsigned{B},\Fsigned{a_1}\},
        \dots,
        \{\Tsigned{B},\Fsigned{a_m}\},
        \{\Tsigned{B},\Tsigned{a_{m+1}}\},
        \dots,
        \{\Tsigned{B},\Tsigned{a_n}\}\,\}
        \]
      }
    \item<only@3->\normalsize
      \alert{%
      \(
      a_1          \wedge \dots \wedge a_m      \wedge
      \neg a_{m+1} \wedge \dots \wedge \neg a_n
      \rightarrow v_B
      \)}

      \smallskip
      gives rise to the nogood
      \[
      \clno{B}
      =
      \alert{\{\Fsigned{B},\Tsigned{a_1},\dots,\Tsigned{a_m},\Fsigned{a_{m+1}},\dots,\Fsigned{a_n}\}}
      \]
      \bigskip
  \end{enumerate}
\end{itemize}
\end{frame}
%------------------------------------------------------------
\begin{frame}{Nogoods from logic programs\\[-1ex]\normalsize via program completion}

  \begin{itemize}
  \item Analogously, the (atom-oriented) equivalence
    \[
    a \leftrightarrow v_{B_1} \vee \dots \vee v_{B_k}
    \]
    yields the nogoods
    \begin{enumerate}
    \item []
    \item
      \(
      \ClNo{a} = \{\,\{\Fsigned{a},\Tsigned{B_1}\},\dots,\{\Fsigned{a},\Tsigned{B_k}\}\,\}
      \)
      and
    \item []
    \item
      \(
      \clno{a} = \{\Tsigned{a},\Fsigned{B_1},\dots,\Fsigned{B_k}\}
      \)
    \item []
    \end{enumerate}
%  \item [] where $\body{a}=\{B_1,\dots,B_k\}$ dor an atom $a\in\atom{P}$
  \end{itemize}
\end{frame}
% ------------------------------------------------------------
\begin{frame}{Nogoods from logic programs\\[-1ex]\normalsize atom-oriented nogoods}
  \begin{itemize}
  \item<1> For an atom $a$ where $\atbody{P}{a}=\{B_1,\dots,B_k\}$,
    we get
    \[
    \{\Tsigned{a},\Fsigned{B_1},\dots,\Fsigned{B_k}\}
    \quad\text{and}\quad
    \{\,\{\Fsigned{a},\Tsigned{B_1}\},\dots,\{\Fsigned{a},\Tsigned{B_k}\}\,\}
    \]
  \item<2-> \structure{Example}
    Given Atom~$x$ with $\body{x}=\{\{y\},\{\naf{z}\}\}$, we obtain
    \arrayrulewidth=.3pt
    \[
    \begin{array}{|rcl|}
      \hline
      \only<9>{\color{red}}\only<13-14>{\color{green}}x&\leftarrow&\only<8-9,13-14>{\color{red}}y\\
      \only<9>{\color{red}}\only<13-14>{\color{green}}x&\leftarrow&\only<8-9>{\color{red}}\only<14>{\color{green}}\naf{z}\\
      \hline
    \end{array}
    \qquad\qquad
    \begin{array}{p{5pt}l}
      &\alert<4>{\{\Tsigned{x},\Fsigned{\{y\}},\Fsigned{\{\naf{z}\}}\}}
      \\[2pt]
      &\{\,\{\Fsigned{x},\Tsigned{\{y\}}\},\{\Fsigned{x},\Tsigned{\{\naf{z}\}}\}\,\}
    \end{array}
    \]
  \item<3-> [] For \alert<4>{nogood}
    \(
    \alert<4>{\{
      \Tsigned{\only<6-7,11-12>{\color{green}}x},
      \Fsigned{\only<6-7,11-12>{\color{red}}\{y\}},
      \Fsigned{\only<6-7,11-12>{\color{red}}\{\naf{z}\}}
      \}},
    \)
    the signed literal
    \begin{itemize}
    \item<5-9> \Fsigned{\only<7,9>{\color{red}}x} is unit-resulting  wrt \alert<5>{assignment}
      \(
      (\Fsigned{\only<5-9>{\color{red}}{\{y\}}},\Fsigned{\only<5-9>{\color{red}}\{\naf{z}\}})
      \)
      and
    \item<10-> \Tsigned{\only<12,14>{\color{green}}\{\naf{z}\}} is unit-resulting  wrt assignment
      \(
      (\Tsigned{\only<10-14>{\color{green}}x},\Fsigned{\only<10-14>{\color{red}}\{y\}})
      \)
    \end{itemize}
  \end{itemize}
\end{frame}
%------------------------------------------------------------
\begin{frame}{Nogoods from logic programs\\[-1ex]\normalsize body-oriented nogoods}
  \begin{itemize}
  \item <1> For a body $B=\{a_1,\dots,a_m,\naf{a_{m+1}},\dots,\naf{a_n}\}$, %\pause,
    we get
    \vspace{-2pt}
    \[
    \begin{array}{l}
      \{\Fsigned{B},\Tsigned{a_1},\dots,\Tsigned{a_m},\Fsigned{a_{m+1}},\dots,\Fsigned{a_n}\}
      \\[2pt]
      \{\,
      \{\Tsigned{B},\Fsigned{a_1}\},
      \dots,
      \{\Tsigned{B},\Fsigned{a_m}\},
      \{\Tsigned{B},\Tsigned{a_{m+1}}\},
      \dots,
      \{\Tsigned{B},\Tsigned{a_n}\}
      \,\}
    \end{array}
  \]
  \item<2-> \structure{Example}
    Given Body~$\{x,\naf{y}\}$, we obtain
    \arrayrulewidth=.3pt
    \[
    \begin{array}{@{}|r@{}l|}
      \hline
      \ldots\leftarrow{}&x,\naf{y}\\[-1.2ex]
      &\quad\vdots\\[-3pt]
      \ldots\leftarrow{}&x,\naf{y}\\
      \hline
    \end{array}
    \quad\quad
    \begin{array}{p{5pt}l}
      &\{\Fsigned{\{x,\naf{y}\}},\Tsigned{x},\Fsigned{y}\}
      \\[2pt]
      &\{\,\{\Tsigned{\{x,\naf{y}\}},\Fsigned{x}\},\{\Tsigned{\{x,\naf{y}\}},\Tsigned{y}\}\,\}
    \end{array}
    \]
  \item<3-> []
    For nogood $\clno{\{x,\naf{y}\}}=\{\Fsigned{\{x,\naf{y}\}},\Tsigned{x},\Fsigned{y}\}$,
    the signed literal
    \begin{itemize}
    \item \Tsigned{\{x,\naf{y}\}} is unit-resulting  wrt assignment $(\Tsigned{x},\Fsigned{y})$
      and
    \item \Tsigned{y}                    is unit-resulting  wrt assignment $(\Fsigned{\{x,\naf{y}\}},\Tsigned{x})$
    \end{itemize}
  \end{itemize}
\end{frame}
%------------------------------------------------------------
\begin{frame}{Characterization of stable models\\\normalsize
  for \alert<3>{tight} logic programs\only<3>{, ie.\ \alert{free of positive recursion}}}
\bigskip
Let $P$ be a logic program and
\[
\begin{array}{lcr@{}c@{}l}
\Delta_P
& = &
% \phantom{\cup\ }
\{\clno{a} \mid a\in\atom{P}\}
& {} \cup {} &
\{\delta\in\ClNo{a} \mid a\in\atom{P}\}
\\[2pt]
& \cup &
\{\clno{B} \mid B\in\body{P}\}
& {} \cup {} &
\{\delta\in\ClNo{B}\mid B\in\body{P}\}
\end{array}
\]
\bigskip
\pause
\begin{theorem}
Let $P$ be a \alert{tight} logic program. Then,

\qquad $X\subseteq\atom{P}$ is a stable model of~$P$ \alert{iff}

\qquad $X=\tlits{\ass}\cap\atom{P}$ for a (unique) solution \ass\ for $\Delta_P$
\end{theorem}
% \pause
% \begin{itemize}
% \item [\ithand] The set $\Delta_P$ of nogoods captures inferences from
% \\  (program $P$ and) program completion
% \end{itemize}
\end{frame}
% %----------------------------------------------------------------
% \begin{frame}{Atom-oriented nogoods and tableau rules}

% \begin{itemize}
% \item Tableau rules FTA, BFA, FFA, and BTA are atom-oriented
% \pause
% \item
%   For an atom $a$ such that $\body{a}=\{B_1,\dots,B_k\}$,\\ consider the equivalence:
%   \qquad
%   \(
%   a \leftrightarrow v_{B_1} \vee \dots \vee v_{B_k}
%   \)
%   \\\mbox{~}
% \pause
% \item Inferences from nogoods \
%   \(
%   \ClNo{a}
%   =
%   \{\,\{\Fsigned{a},\Tsigned{B_1}\},\dots,\{\Fsigned{a},\Tsigned{B_k}\}\,\}
%   \)\\
%   correspond to those from tableau rules \alert{FTA} and \alert{BFA}:
% \[
% \begin{array}{c}
% a \leftarrow B \\
% \Tsigned{B} \\\hline
% \Tsigned{a}
% \end{array}
% \qquad\qquad%\qquad
% \begin{array}{c}
% a \leftarrow B \\
% \Fsigned{a} \\\hline
% \Fsigned{B}
% \end{array}
% \]
% % \pause
% %   \begin{description}
% %   \item [FTA:]
% %     \(
% %     (v_{B_1} \vee \dots \vee v_{B_m}) \rightarrow a
% %     \)
% %   \item[BFA:]
% %     \(
% %     \neg{a} \rightarrow (\neg{v_{B_1}} \wedge \dots \wedge \neg{v_{B_m}})
% %     \)
% %   \end{description}
% %   both being equivalent to
% %   \(
% %   (\neg{v_{B_1}} \vee a) \wedge \dots \wedge (\neg{v_{B_m}} \vee a)
% %   \)
% %\newslide\mbox{}\bigskip
% % \newpage
% \pause
% \item Inferences from nogood
%   \(
%   \clno{a}
%   =
%   \{\Tsigned{a},\Fsigned{B_1},\dots,\Fsigned{B_k}\}
%   \)\\
%   correspond to those from tableau rules \alert{FFA} and \alert{BTA}:
% \[
% \begin{array}[b]{c}
% \Fsigned{B_1},\dots,\Fsigned{B_k} % &
% % \raisebox{-8pt}[0pt][0pt]{($\body{a} = \{B_1,\dots,B_k\}$)}
% \\\hline%{1-1}
% \Fsigned{a}
% \end{array}
% \qquad\qquad%\qquad
% \begin{array}[b]{c}
% \Tsigned{a} \\
% \Fsigned{B_1},\dots,\Fsigned{B_{i-1}},\Fsigned{B_{i+1}},\dots,\Fsigned{B_k} % &
% % \raisebox{-8pt}[0pt][0pt]{($\body{a} = \{B_1,\dots,B_k\}$)}
% \\\hline%{1-1}
% \Tsigned{B_i}
% \end{array}
% \]
% % \begin{description}
% % \item[FFA:]
% %     \(
% %     (\neg{v_{B_1}} \wedge \dots \wedge \neg{v_{B_m}}) \rightarrow \neg{a}
% %     \)
% %   \item[BTA:]
% %     \(
% %     a \rightarrow (v_{B_1} \vee \dots \vee v_{B_m})
% %     \)
% %   \end{description}
% %   both being equivalent to
% %   \(
% %   v_{B_1} \vee \dots \vee v_{B_m} \vee \neg{a}
% %   \)
% \end{itemize}
% \end{frame}
% %----------------------------------------------------------------
% \begin{frame}{Body-oriented nogoods and tableau rules}

% \begin{itemize}
% \item Tableau rules FTB, BFB, FFB, and BTB are body-oriented
% \pause
% \item For a body
%   $B=\{a_1,\dots,a_m,\naf{a_{m+1}},\dots,\naf{a_n}\}=\{l_1,\dots,l_n\}$,\\
%   consider the equivalence:
%   \quad
%   \(
%   v_{B}\leftrightarrow a_1\wedge\dots\wedge a_m\wedge \neg a_{m+1}\wedge\dots\wedge\neg a_n
%   \)
% \pause
% \item Inferences from nogood
%   \(
%   \clno{B}
%   =
%   \{\Fsigned{B},\Tsigned{a_1},\dots,\Tsigned{a_m},\Fsigned{a_{m+1}},\dots,\Fsigned{a_n}\}
%   \)\\
%   correspond to those from tableau rules \alert{FTB} and \alert{BFB}:
% \[
% \begin{array}{c}
% a \leftarrow l_1,\dots,l_n \\
% \plit{l_1},\dots,\plit{l_n} \\\hline
% \Tsigned{\{l_1,\dots,l_n\}}
% \end{array}
% \qquad\qquad%\qquad
% \begin{array}{c}
% \Fsigned{\{l_1,\dots,l_n\}} \\
% \plit{l_1},\dots,\plit{l_{i-1}},\plit{l_{i+1}},\dots,\plit{l_n} \\\hline
% \nlit{l_i}
% \end{array}
% \]
% % \begin{description}
% %   \item [FTB:]
% %     \(
% %     (a_1\wedge\dots\wedge a_m\wedge\neg a_{m+1}\wedge\dots\wedge\neg a_n)\rightarrow v_{B}
% %     \)
% %   \item[BFB:]
% %     \(
% %     \neg{v_{B}} \rightarrow
% %     (\neg a_1\vee\dots\vee\neg a_m\vee a_{m+1}\vee\dots\vee a_n)
% %     \)
% %   \end{description}
% %   both being equivalent to
% %   \(
% %   (\neg a_1\vee\dots\vee\neg a_m\vee a_{m+1}\vee\dots\vee a_n
% %   \vee v_{B})
% %   \)
% %\newslide\mbox{}\bigskip
% % \newpage
% \pause
% \item Inferences from nogoods
%   \(
%   \ClNo{B}
%   =
%   \{\,\{\Tsigned{B},\Fsigned{a_1}\},
%   \dots,
%   \{\Tsigned{B},\Fsigned{a_m}\},
%   \{\Tsigned{B},\Tsigned{a_{m+1}}\},
%   \dots,
%   \{\Tsigned{B},\Tsigned{a_n}\}\,\}
%   \)
%   correspond to those from tableau rules \alert{FFB} and \alert{BTB}:
% \begin{array}[b]{c}
% a \leftarrow l_1,\dots,l_i,\dots,l_n \\
% \nlit{l_i} \\\hline
% \Fsigned{\{l_1,\dots,l_i,\dots,l_n\}}
% \end{array}
% \qquad\qquad%\qquad
% \begin{array}[b]{c}
% \Tsigned{\{l_1,\dots,l_i,\dots,l_n\}} \\\hline
% \plit{l_i}
% \end{array}
% %   \begin{description}
% %   \item [FFB:]
% %     \(
% %     (\neg a_1\vee\dots\vee\neg a_m\vee a_{m+1}\vee\dots\vee a_n)
% %     \rightarrow \neg{v_{B}}
% %     \)
% %   \item [BTB:]
% %     \(
% %     v_{B}\rightarrow (a_1\wedge\dots\wedge a_m\wedge\neg a_{m+1}\wedge\dots\wedge\neg a_n)
% %     \)
% %   \end{description}
% %   both being equivalent to
% %     \(
% %     (a_1 \vee \neg{v_{B}})
% %     \wedge \dots \wedge
% %     (a_m \vee \neg{v_{B}})
% %     \wedge
% %     (\neg a_{m+1} \vee \neg{v_{B}})
% %     \wedge \dots \wedge
% %     (\neg a_n \vee \neg{v_{B}})
% %     \)
% \end{itemize}
% \end{frame}
%------------------------------------------------------------
\subsection{Nogoods from loop formulas}
%----------------------------------------------------------------
\begin{frame}{Nogoods from logic programs\\[-1ex]\normalsize via loop formulas}

  Let $P$ be a normal logic program and recall that:
  %
  \begin{itemize}
  \item <1->
    For $L\subseteq\atom{P}$,
    the external supports of $L$ for $P$ are
    \begin{eqnarray*}
      \ES{L}{P}
      & = &
      \{r\in P\mid\head{r}\in L \text{ and } \pbody{r}\cap L=\emptyset\}
    \end{eqnarray*}

  \item <2->
    The (disjunctive) loop formula of $L$ for $P$ is
    \begin{eqnarray*}
      \LFM{L}{P}
      & = &
      \big(
      \mbox{$\bigvee_{A\in L}$} A
      \big)
      \rightarrow
      \big(
      \mbox{$\bigvee_{r\in\ES{L}{P}}$}
      % \comp BUGGED :)
      {\body{r}}
      \big)
      \\
      & \leftrightarrow &
      \big(
      % \left(
      \mbox{$\bigwedge_{r\in\ES{L}{P}}$}
      \neg% \comp
      {\body{r}}
      \big)
      % \right)
      \rightarrow
      \big(
      \mbox{$\bigwedge_{A\in L}$}\neg A
      \big)
    \end{eqnarray*}
    \vspace*{-4mm}
    \begin{itemize}
    \item \structure{Note}
      The loop formula of $L$ enforces
      all atoms in~$L$ to be \emph{false}
      \par
      whenever $L$ is not externally supported
    \end{itemize}

  \item <3-> The external bodies of $L$ for $P$ are
    \begin{eqnarray*}
      \EB{L}{P} & = & \{\body{r} \mid r\in \ES{L}{P}\}
    \end{eqnarray*}
  \end{itemize}
\end{frame}
%------------------------------------------------------------
\begin{frame}{Nogoods from logic programs\\[-1ex]\normalsize loop nogoods} %\label{cdnl:lambda}
  \begin{itemize}
  \item<1->
    For a logic program $P$ and some $\emptyset \subset U\subseteq\atom{P}$,\\
    define the \alert{loop nogood} of an atom $a\in U$ as
    \begin{eqnarray*}
      \lambda(a,U)
      & = &
      \{\Tsigned{a},\Fsigned{B_1},\dots,\Fsigned{B_k}\}
    \end{eqnarray*}
    where $\EB{U}{P}=\{B_1,\dots,B_k\}$
    \medskip
  \item<2->
    We get the following set of loop nogoods for $P$:
    \begin{eqnarray*}
      \Lambda_P
      & = &
      \textstyle
      \bigcup_{\emptyset\subset U\subseteq \atom{P}}
      \{
      \lambda(a,U)
      \mid
      a\in U
      \}
    \end{eqnarray*}
  \item<3->  The set $\Lambda_P$ of loop nogoods denies cyclic
    support among \emph{true} atoms
  \end{itemize}
\end{frame}
%------------------------------------------------------------
\begin{frame}{Example}
  \begin{itemize}
  \item<1-> Consider the program
    \[
    \left\{
      \begin{array}{l}
        x  \leftarrow  \naf{y}\\
        y  \leftarrow  \naf{x}
      \end{array}
      \
      \begin{array}{l}
        u  \leftarrow  x\\
        \alert{u}  \leftarrow  \alert{v}\\
        \alert{v}  \leftarrow  \alert{u}, y
      \end{array}
    \right\}
    \]
  \item<2-> For $u$ in the set $\{u,v\}$, we obtain the loop nogood:
    \begin{eqnarray*}
      \lambda(u,\{u,v\}) &=& \{\Tsigned{u},\Fsigned{\{x\}}\}
    \end{eqnarray*}
  \item<3-> [] Similarly for $v$ in $\{u,v\}$, we get:
    \begin{eqnarray*}
      \lambda(v,\{u,v\}) &=& \{\Tsigned{v},\Fsigned{\{x\}}\}
    \end{eqnarray*}
  \end{itemize}
\end{frame}
% ------------------------------------------------------------
\begin{frame}{Characterization of stable models}
\bigskip
\begin{theorem}
Let $P$ be a logic program. Then,

\qquad $X\subseteq\atom{P}$ is a stable model of~$P$ \alert{iff}

\qquad $X=\tlits{\ass}\cap\atom{P}$ for a (unique)
solution \ass\ for $\Delta_P\cup\Lambda_P$
\end{theorem}
\bigskip
\pause
\begin{description}
\item [Some remarks] \
\begin{itemize}
\item
Nogoods in $\Lambda_P$ augment $\Delta_P$
with conditions checking for \alert{unfounded sets},
in particular,
those being loops
\item
While $|\Delta_P|$ is linear in the size of $P$,
$\Lambda_P$ may contain \alert{exponentially many} (non-redundant) loop nogoods
\end{itemize}
\end{description}
\end{frame}
% ----------------------------------------------------------------------
\section{Conflict-driven nogood learning}
% ----------------------------------------------------------------------
% ----------------------------------------------------------------------
\begin{frame}{Towards conflict-driven search}
  \medskip
  Boolean constraint solving algorithms pioneered for SAT led to
  \smallskip
  \begin{itemize}
  \item<1-> \structure{Traditional DPLL-style approach}
  \item<1-> [] \structure{\footnotesize (DPLL stands for `Davis-Putnam-Logemann-Loveland')}
    \begin{itemize}\normalsize
    \item Unit propagation
    \item Backtracking
    \smallskip
    \item in ASP, eg \smodels\
    \end{itemize}
  \medskip
  \item<1-> \structure{Modern CDCL-style approach}
  \item<1-> [] \structure{\footnotesize (CDCL stands for `Conflict-Driven Constraint Learning')}
    \begin{itemize}\normalsize
    \item Unit propagation
    \item Conflict analysis (via resolution)
    \item Learning + Backjumping + Assertion
    \smallskip
    \item in ASP, eg \clasp\
    \end{itemize}
  \end{itemize}
\end{frame}
% ----------------------------------------------------------------------
%
%%% Local Variables:
%%% mode: latex
%%% TeX-master: "../../main"
%%% End:

% ----------------------------------------------------------------------
\begin{frame}{DPLL-style solving}
  \bigskip
  \bigskip
  \textbf{loop}
  \begin{itemize}
  \item [] \textit{propagate}     \hfill\structure{// deterministically assign literals}
  \item [] \textbf{if} no conflict \textbf{then}
    \begin{itemize}\normalsize
    \item [] \textbf{if} all variables assigned
      \textbf{then}
      \textbf{return} solution
    \item [] \textbf{else}
      \textit{decide}
                                  \hfill\structure{// non-deterministically assign some literal}
    \end{itemize}
  \item [] \textbf{else}
    \begin{itemize}\normalsize
    \item [] \textbf{if} top-level conflict
      \textbf{then}
      \textbf{return} unsatisfiable
    \item [] \textbf{else}
      \begin{itemize}\normalsize
      \item [] \textit{backtrack} \hfill\structure{// unassign literals propagated after last decision}
      \item [] \textit{flip}      \hfill\structure{// assign complement of last decision literal}
      \end{itemize}
    \end{itemize}
  \end{itemize}
\end{frame}
% ----------------------------------------------------------------------
%
%%% Local Variables:
%%% mode: latex
%%% TeX-master: "../../main"
%%% End:

\begin{frame}{CDCL-style solving}
  \bigskip
  \bigskip
  \textbf{loop}
  \begin{itemize}
  \item [] \textit{propagate}      \hfill// deterministically assign literals
  \item [] \textbf{if} no conflict \textbf{then}
    \begin{itemize}
    \item [] \textbf{if} all variables assigned
      \textbf{then}
      \textbf{return} solution
    \item [] \textbf{else}
      \textit{decide}
      \hfill// non-deterministically assign some literal
    \end{itemize}
  \item [] \textbf{else}
    \begin{itemize}
    \item [] \textbf{if} top-level conflict
      \textbf{then}
      \textbf{return} unsatisfiable
    \item [] \textbf{else}
      \begin{itemize}
      \item [] \textit{analyze} \hfill// analyze conflict and add conflict constraint
      \item [] \textit{backjump} \hfill// unassign literals until conflict constraint is unit
      \end{itemize}
    \end{itemize}
  \end{itemize}
\end{frame}
%
%%% Local Variables:
%%% mode: latex
%%% TeX-master: "../asp"
%%% End:

% ----------------------------------------------------------------------
\subsection{CDNL-ASP Algorithm}
% ----------------------------------------------------------------------
\begin{frame}{Outline of CDNL-ASP algorithm}
\begin{itemize}
\item Keep track of deterministic consequences by unit propagation on:
  \begin{itemize}
  \item Program completion \hfill [$\Delta_P$]%
  \item Loop nogoods, determined and recorded on demand \hfill [$\Lambda_P$]%
    % \begin{itemize}
    % \item \structure{Note} Dedicated unfounded set detection !
    % \end{itemize}
  \item Dynamic nogoods, derived from conflicts and unfounded sets \hfill[$\nabla$]%
  \end{itemize}
\pause
\item When a nogood in $\Delta_P\cup\nabla$ becomes \alert<2>{violated}:
  \begin{itemize}
  \item \alert<2>{Analyze} the conflict by resolution
    \par
    (until reaching a {Unique Implication Point}, short: UIP)
  \item \alert<2>{Learn} the derived conflict nogood~$\delta$
  \item \alert<2>{Backjump} to the earliest (heuristic) choice such that
        the\par complement of the UIP is unit-resulting for $\delta$
  \item \alert<2>{Assert} the complement of the UIP and proceed\\
        (by unit propagation)
  \end{itemize}
\pause
\item Terminate when either:
  \begin{itemize}
  \item Finding a stable model (a solution for $\Delta_P\cup\Lambda_P$)
  \item Deriving a conflict independently of (heuristic) choices
  \end{itemize}
\end{itemize}
\end{frame}
%------------------------------------------------------------
\begin{frame}[c]%
\SetAlFnt{\scriptsize}
\input{solving/algo_cdnl}
\end{frame}
%------------------------------------------------------------
\begin{frame}{Observations}
\begin{itemize}
\item Decision level~$\mathit{dl}$,
      initially set to~$0$,
      is used to count the number of heuristically chosen
      literals in assignment~\ass
\item For a heuristically chosen literal
      $\sigma_{\!d}=\Tsigned{a}$ or $\sigma_{\!d}=\Fsigned{a}$,
      respectively,
      we require $a\in(\atom{P}\cup\body{P})\setminus(\tlits{\ass}\cup\flits{\ass})$
\item For any literal $\sigma\in\ass$,
      $\dlevel{\sigma}$ denotes the decision level of~$\sigma$,
      viz.\ the value $\mathit{dl}$ had when $\sigma$ was assigned
\pause
\item A conflict is detected from violation of a nogood $\varepsilon\subseteq\Delta_P\cup\nabla$
\item A conflict at decision level~$0$
      (where~\ass\ contains no heuristically chosen literals)
      indicates non-existence of stable models
\item A nogood $\delta$ derived by conflict analysis is
      \alert{asserting}, that is,\\
      some literal is unit-resulting for $\delta$ at a decision level $k<\mathit{dl}$
\pause
  \begin{itemize}
  \item After learning~$\delta$ and backjumping to
                  decision level~$k$,\\ at least one literal is
                  newly derivable by unit propagation
  \item No explicit flipping of heuristically chosen literals !
  \end{itemize}
\end{itemize}
\end{frame}
%------------------------------------------------------------
\begin{frame}{Example: {CDNL-ASP}}
Consider
\begin{eqnarray*}
P
& = &
\left\{
  \begin{array}{l}
x  \leftarrow  \naf{y}\\%[-3pt]
y  \leftarrow  \naf{x}
\end{array}
\
\begin{array}{l}
u  \leftarrow x,y\\%[-3pt]
u  \leftarrow v%\\%[-3pt]
\end{array}
\
\begin{array}{l}
v  \leftarrow x\\%[-3pt]
v  \leftarrow u,y%\\%[-3pt]
\end{array}
\
\begin{array}{l}
w  \leftarrow \naf{x},\naf{y}\\
\mbox{~}
\end{array}
\right\}
\end{eqnarray*}
%\pause
\[
\begin{array}[b]{|c|ll|l|}
\hline
\mathit{dl}& \sigma_{\!d} & \overline{\sigma} & \delta
\\\hline\hline
\onslide<2->{1} & \onslide<2->{\Tsigned{u}} & &
\\\hline
\onslide<3->{2} & \onslide<3->{\Fsigned{\{\naf{x},\naf{y}\}}} & &
\\
  & & \onslide<4->{\Fsigned{w}}     & \onslide<4->{\{\Tsigned{w},\Fsigned{\{\naf{x},\naf{y}\}}\}=\clno{w}}
\\\hline
\onslide<5->{3} & \onslide<5->{\Fsigned{\{\naf{y}\}}} & &
\\
  & & \onslide<6->{\Fsigned{x}} & \onslide<6->{\{\Tsigned{x},\Fsigned{\{\naf{y}\}}\}=\clno{x}}
\\
  & & \onslide<6->{\Fsigned{\{x\}}} & \onslide<6->{\{\Tsigned{\{x\}},\Fsigned{x}\}\in\ClNo{\{x\}}}
\\
  & & \onslide<6->{\Fsigned{\{x,y\}}} & \onslide<6->{\{\Tsigned{\{x,y\}},\Fsigned{x}\}\in\ClNo{\{x,y\}}}
\\
  & & \onslide<7->{\vdots} & \onslide<7->{\vdots}
\\
& & & \onslide<7->{\{\Tsigned{u},\Fsigned{\{x\}},\Fsigned{\{x,y\}}\}=\lambda(u,\{u,v\})}
\\\hline
\end{array}
~\raisebox{1mm}{\onslide<7->{\KO}}
\]
\end{frame}
%------------------------------------------------------------
\begin{frame}{Example: {CDNL-ASP}}
Consider
\begin{eqnarray*}
P
& = &
\left\{
  \begin{array}{l}
x  \leftarrow  \naf{y}\\%[-3pt]
y  \leftarrow  \naf{x}
\end{array}
\
\begin{array}{l}
u  \leftarrow x,y\\%[-3pt]
u  \leftarrow v%\\%[-3pt]
\end{array}
\
\begin{array}{l}
v  \leftarrow x\\%[-3pt]
v  \leftarrow u,y%\\%[-3pt]
\end{array}
\
\begin{array}{l}
w  \leftarrow \naf{x},\naf{y}\\
\mbox{~}
\end{array}
\right\}
\end{eqnarray*}
%
\[
\begin{array}[b]{|c|ll|l|}
\hline
\mathit{dl}& \sigma_{\!d} & \overline{\sigma} & \delta
\\\hline\hline
1 & \alert<4>{\Tsigned{u}} & \phantom{\Fsigned{\{x,y\}}}&\phantom{\{\Tsigned{u},\Fsigned{\{x\}},\Fsigned{\{x,y\}}\}=\lambda(u,\{u,v\})}
\\
  & \phantom{\Fsigned{\{\naf{x},\naf{y}\}}}& \onslide<2->{\alert<4>{\Tsigned{x}}} & \onslide<2->{\{\Tsigned{u},\Fsigned{x}\}\in\nabla}
\\
  & & \onslide<3->{\vdots} & \onslide<3->{\vdots}
\\
  & & \onslide<3->{\alert<4>{\Tsigned{v}}} & \onslide<3->{\{\Fsigned{v},\Tsigned{\{x\}}\}\in\ClNo{v}}
\\
  & & \onslide<3->{\Fsigned{y}} & \onslide<3->{\{\Tsigned{y},\Fsigned{\{\naf{x}\}}\}=\clno{y}}
\\
  & & \onslide<3->{\Fsigned{w}} & \onslide<3->{\{\Tsigned{w},\Fsigned{\{\naf{x},\naf{y}\}}\}=\clno{w}}
\\\hline
\end{array}
\]
\bigskip\bigskip\bigskip\bigskip\bigskip\bigskip
\end{frame}
%------------------------------------------------------------
\subsection{Nogood Propagation}
%------------------------------------------------------------
\begin{frame}{Outline of {NogoodPropagation}}
\begin{itemize}
\item Derive deterministic consequences via:
  \begin{itemize}
  \item Unit propagation on $\Delta_P$ and $\nabla$;
  \item Unfounded sets~$U\subseteq\atom{P}$
  \end{itemize}
\item Note that $U$ is \alert{unfounded} if $\EB{U}{P}\subseteq\flits{\ass}$
  \begin{itemize}
  \item \structure{Note} For any $a\in U$, we have $(\lambda(a,U)\setminus\{\Tsigned{a}\})\subseteq\ass$
  \end{itemize}
\pause
\item An ``interesting'' unfounded set~$U$ satisfies:
\[
  \emptyset\subset U \subseteq (\atom{P}\setminus\flits{\ass})
\]
\item Wrt a fixpoint of unit propagation,\\
\pause
      such an unfounded set contains some loop of~$P$
  \begin{itemize}
  \item \structure{Note} Tight programs do not yield ``interesting'' unfounded sets !
  \end{itemize}
\pause
\item Given an unfounded set $U$ and some $a\in U$,
      adding $\lambda(a,U)$ to $\nabla$ triggers a conflict or
      further derivations by unit propagation
  \begin{itemize}
  \item \structure{Note} Add loop nogoods atom by atom to eventually falsify all $a\in U$
  \end{itemize}
\end{itemize}
\end{frame}
%------------------------------------------------------------
\begin{frame}[c]
\SetAlFnt{\tiny}
\input{solving/algo_unit_propagation}
\end{frame}
%------------------------------------------------------------
\begin{frame}{Requirements for \UnFoundedSet}
\begin{itemize}
\item<1-> Implementations of \UnFoundedSet\ must guarantee the following for a result $U$
  \begin{enumerate}
  \item $U\subseteq (\atom{P}\setminus\flits{\ass})$
  \item $\EB{U}{P}\subseteq\flits{\ass}$
  \item $U=\emptyset$ iff there is no nonempty unfounded subset of $(\atom{P}\setminus\flits{\ass})$
  \end{enumerate}

\item<2-> Beyond that, there are various alternatives, such as:
  \begin{itemize}
  \item Calculating the greatest unfounded set
  \item Calculating unfounded sets within strongly connected components
        of the positive atom dependency graph of~$P$
        \medskip
  \item Usually, the latter option is implemented in ASP solvers
  \end{itemize}
\end{itemize}
\end{frame}
%------------------------------------------------------------
\begin{frame}{Example: {NogoodPropagation}}
Consider
\begin{eqnarray*}
P
& = &
\left\{
  \begin{array}{l}
x  \leftarrow  \naf{y}\\%[-3pt]
y  \leftarrow  \naf{x}
\end{array}
\
\begin{array}{l}
u  \leftarrow x,y\\%[-3pt]
u  \leftarrow v%\\%[-3pt]
\end{array}
\
\begin{array}{l}
v  \leftarrow x\\%[-3pt]
v  \leftarrow u,y%\\%[-3pt]
\end{array}
\
\begin{array}{l}
w  \leftarrow \naf{x},\naf{y}\\
\mbox{~}
\end{array}
\right\}
\end{eqnarray*}
%
\[
\footnotesize
\begin{array}[b]{|c|ll|l|}
\hline
\mathit{dl}& \sigma_{\!d} & \overline{\sigma} & \delta
\\\hline\hline
1 & \Tsigned{u} & &
\\\hline
2 & \Fsigned{\{\naf{x},\naf{y}\}} & &
\\
  & & \Fsigned{w}     & \{\Tsigned{w},\Fsigned{\{\naf{x},\naf{y}\}}\}=\clno{w}
\\\hline
3 & \Fsigned{\{\naf{y}\}} & &
\\
  & & \Fsigned{x}     & \{\Tsigned{x},\Fsigned{\{\naf{y}\}}\}=\clno{x}
\\
  & & \Fsigned{\{x\}} & \{\Tsigned{\{x\}},\Fsigned{x}\}\in\ClNo{\{x\}}
\\
  & & \Fsigned{\{x,y\}} & \{\Tsigned{\{x,y\}},\Fsigned{x}\}\in\ClNo{\{x,y\}}
\\
  & & \Tsigned{\{\naf{x}\}} & \{\Fsigned{\{\naf{x}\}},\Fsigned{x}\}=\clno{\{\naf{x}\}}
\\
  & & \Tsigned{y} & \{\Fsigned{\{\naf{y}\}},\Fsigned{y}\}=\clno{\{\naf{y}\}}
\\
%   & & \Fsigned{\{\naf{x},\naf{y}\}} & \{\Tsigned{\{\naf{x},\naf{y}\}},\Tsigned{y}\}\in\ClNo{\{\naf{x},\naf{y}\}}
% \\
  & & \Tsigned{\{v\}} & \{\Tsigned{u},\Fsigned{\{x,y\}},\Fsigned\{v\}\}=\clno{u}
\\
  & & \Tsigned{\{u,y\}} & \{\Fsigned{\{u,y\}},\Tsigned{u},\Tsigned{y}\}=\clno{\{u,y\}}
\\
  & & \Tsigned{v} & \{\Fsigned{v},\Tsigned{\{u,y\}}\}\in\ClNo{v}
\\
  & & & \{\Tsigned{u},\Fsigned{\{x\}},\Fsigned{\{x,y\}}\}=\lambda(u,\{u,v\})
\\\hline
\end{array}
~\raisebox{1mm}{\KO}
\]
\end{frame}
%------------------------------------------------------------
\subsection{Conflict Analysis}
%------------------------------------------------------------
\begin{frame}{Outline of {ConflictAnalysis}}
\begin{itemize}
\item Conflict analysis is triggered whenever
      some nogood $\delta\in\Delta_P\cup\nabla$
      becomes violated, viz.\ $\delta\subseteq\ass$,
      at a decision level~$\mathit{dl}>0$
      \begin{itemize}
      \item Note that all but the first literal assigned at $\mathit{dl}$
        have been unit-resulting for nogoods
        $\varepsilon\in\Delta_P\cup\nabla$
      \item If $\sigma\in\delta$ has been unit-resulting for~$\varepsilon$,
        we obtain a new violated nogood
        by resolving $\delta$ and $\varepsilon$ as follows:
        \[
        (\delta\setminus\{\sigma\})\cup(\varepsilon\setminus\{\overline{\sigma}\})
        \]
      \end{itemize}
\pause
\item Resolution is directed by resolving first over the literal $\sigma\in\delta$
      derived last, viz.\ $(\delta\setminus\ass[\sigma])=\{\sigma\}$
      \begin{itemize}
      \item Iterated resolution progresses in inverse
                     order of assignment
      \end{itemize}
\pause
\item Iterated resolution stops as soon as it generates
      a nogood~$\delta$ containing exactly one literal~$\sigma$ assigned
      at decision level~$\mathit{dl}$
  \begin{itemize}
  \item This literal $\sigma$ is called \alert{First Unique Implication Point} (First-UIP)
  \item All literals in $(\delta\setminus\{\sigma\})$ are assigned
        at decision levels smaller than~$\mathit{dl}$
  \end{itemize}
\end{itemize}
\end{frame}
%------------------------------------------------------------
\begin{frame}[c]
\begin{algorithm}[H]
\caption{\textsc{ConflictAnalysis}\label{algo:conflict:analysis}}
\BlankLine
\Input{A non-empty violated nogood $\delta$, a normal program $P$, a set $\nabla$ of nogoods,
       and an % ordered
       assignment \ass.}
\Output{A derived nogood and a decision level.}
\BlankLine
\Loop{}{%
  \Let{$\sigma\in\delta$ \SuchThat $\delta\setminus\prefix{\ass}{\sigma} = \{\sigma\}$}{%
    $k:=\max(\{\dlevel{\rho}\mid\rho\in\delta\setminus\{\sigma\}\}\cup\{0\})$\;
    \uIf{$k=\dlevel{\sigma}$}{%
      \Let{$\varepsilon\in\CN{P}\cup\nabla$ \SuchThat $\varepsilon\setminus\prefix{\ass}{\sigma} = \{\opp{\sigma}\}$}{%
        $\delta:=(\delta\setminus\{\sigma\})\cup(\varepsilon\setminus\{\opp{\sigma}\})$\AlgoComm*{resolution}
      }
    }
    \lElse{%
      \Return $(\delta,k)$
    }
  }
}
\end{algorithm}

%%% Local Variables:
%%% mode: latex
%%% TeX-master: "../../../main"
%%% End:

\end{frame}
%------------------------------------------------------------
\begin{frame}{Example: {ConflictAnalysis}}
Consider
\begin{eqnarray*}
P
& = &
\left\{
  \begin{array}{l}
x  \leftarrow  \naf{y}\\%[-3pt]
y  \leftarrow  \naf{x}
\end{array}
\
\begin{array}{l}
u  \leftarrow x,y\\%[-3pt]
u  \leftarrow v%\\%[-3pt]
\end{array}
\
\begin{array}{l}
v  \leftarrow x\\%[-3pt]
v  \leftarrow u,y%\\%[-3pt]
\end{array}
\
\begin{array}{l}
w  \leftarrow \naf{x},\naf{y}\\
\mbox{~}
\end{array}
\right\}
\end{eqnarray*}
%
\[
\footnotesize
\begin{array}{|@{\:}c@{\:}|@{\:}l@{\:}l@{\:}|@{\:}l@{\:}|l}
\cline{1-4}
\mathit{dl}& \sigma_{\!d} & \overline{\sigma} & \delta
\\\cline{1-4}\cline{1-4}
1 & \alert<2,5,8>{\Tsigned{u}} & &
\\\cline{1-4}
2 & \Fsigned{\{\naf{x},\naf{y}\}} & &
\\
  & & \Fsigned{w}     & \{\Tsigned{w},\Fsigned{\{\naf{x},\naf{y}\}}\}=\clno{w}
\\\cline{1-4}
3 & \Fsigned{\{\naf{y}\}} & &
\\
  & & \alert<5-6,8-9>{\Fsigned{x}}     & \{\Tsigned{x},\Fsigned{\{\naf{y}\}}\}=\clno{x}
\\
  & & \alert<2-3,5-6>{\Fsigned{\{x\}}} & \{\alert<7>{\Tsigned{\{x\}}},\alert<8>{\Fsigned{x}}\}\in\ClNo{\{x\}}
& \onslide<8->{\alert<8,10>{\{\Tsigned{u},\alert<9>{\Fsigned{x}}\}}}
\\
  & & \alert<2-3>{\Fsigned{\{x,y\}}} & \{\alert<4>{\Tsigned{\{x,y\}}},\alert<5>{\Fsigned{x}}\}\in\ClNo{\{x,y\}} & \onslide<5->{\alert<5>{\{\alert<8>{\Tsigned{u}},\alert<6,8>{\Fsigned{x}},\alert<6-7>{\Fsigned{\{x\}}}\}}}
\\
  & & \Tsigned{\{\naf{x}\}} & \{\Fsigned{\{\naf{x}\}},\Fsigned{x}\}=\clno{\{\naf{x}\}}
\\
  & & \Tsigned{y} & \{\Fsigned{\{\naf{y}\}},\Fsigned{y}\}=\clno{\{\naf{y}\}}
\\
%   & & \Fsigned{\{\naf{x},\naf{y}\}} & \{\Tsigned{\{\naf{x},\naf{y}\}},\Tsigned{y}\}\in\ClNo{\{\naf{x},\naf{y}\}}
% \\
  & & \Tsigned{\{v\}} & \{\Tsigned{u},\Fsigned{\{x,y\}},\Fsigned\{v\}\}=\clno{u}
\\
  & & \Tsigned{\{u,y\}} & \{\Fsigned{\{u,y\}},\Tsigned{u},\Tsigned{y}\}=\clno{\{u,y\}}
\\
  & & \Tsigned{v} & \{\Fsigned{v},\Tsigned{\{u,y\}}\}\in\ClNo{v}
\\
  & & & \alert<2>{\{\alert<5>{\Tsigned{u}},\alert<3,5>{\Fsigned{\{x\}}},\alert<3-4>{\Fsigned{\{x,y\}}}\}}=\lambda(u,\{u,v\}) & \text{\KO}
\\\cline{1-4}
\end{array}
\]
\end{frame}
%------------------------------------------------------------
\begin{frame}{Remarks}
\begin{itemize}
\item
There always is a First-UIP at which conflict analysis terminates
\begin{itemize}
\item
In the worst, resolution stops at the
heuristically chosen literal\\ assigned at decision level~$\mathit{dl}$
\end{itemize}
\pause
\item
The nogood~$\delta$ containing First-UIP~$\sigma$ is violated by~\ass,
viz.\ $\delta\subseteq\ass$
\item
We have $k=\mathit{max}(\{\mathit{dl}(\sigmaAUX)\mid\sigmaAUX\in\delta\setminus\{\sigma\}\}\cup\{0\})<\mathit{dl}$
\pause
  \begin{itemize}
  \item
  After recording $\delta$ in $\nabla$ and backjumping to decision level~$k$,\\
  $\overline{\sigma}$ is unit-resulting for $\delta$ !
  \item
  Such a nogood $\delta$ is called \alert{asserting}
  \end{itemize}
\pause
\item
Asserting nogoods direct conflict-driven search into a different
region of the search space than traversed before,\\
without explicitly flipping any heuristically chosen literal !
\end{itemize}
\end{frame}
% %------------------------------------------------------------
% \section{Implementation via clasp}
% %------------------------------------------------------------
% \begin{frame}{The \texttt{clasp} system}
% \begin{itemize}
% \item Native ASP solver combining conflict-driven search
%       with sophisticated reasoning techniques:
%   \begin{itemize}
%   \item
%   Advanced preprocessing including, e.g.,
%   % well-founded model computation and
%   equivalence reasoning
%   \item
%   Lookback-based decision heuristics
%   \item
%   Restart policies
%   \item
%   Nogood deletion
%   \item
%   Progress saving
%   \item
%   Dedicated data structures for binary and ternary nogoods
%   \item
%   Lazy data structures (watched literals) for long nogoods
%   \item
%   Dedicated data structures for cardinality and weight constraints
%   \item
%   Lazy unfounded set checking based on ``source pointers''
%   \item
%   Tight integration of unit propagation and unfounded set checking
%   \item
%   \alert{Reasoning modes}
%   \item
%   \ldots
%   \end{itemize}
% \pause
% \item[\ithand] Many of these techniques are configurable !
% \end{itemize}
% \end{frame}
% %------------------------------------------------------------
% \begin{frame}{Reasoning modes of \texttt{clasp}}
% Beyond deciding stable model existence,
% \texttt{clasp} allows for:
% \begin{itemize}
% \item Optimization
% \item Enumeration \hfill [without solution recording]
% \item Projective Enumeration \hfill [without solution recording]
% \item Brave and Cautious Reasoning determining the
%   \begin{itemize}
%   \item union or
%   \item intersection
%   \end{itemize}
%       of all stable models by computing only linearly many of them
% \item[\ithand] Reasoning applicable wrt stable models as well as
%                supported models
% \end{itemize}
% \pause
% Front-ends also admit \texttt{clasp} to solve:
% \begin{itemize}
% \item Propositional CNF formulas
% \item Pseudo-Boolean formulas
% \end{itemize}
% \pause
% Find \texttt{clasp} at: \url{http://potassco.org}
% \end{frame}
% ------------------------------------------------------------
% \begin{frame}{The system architecture of \clasp}
% \centering
% \ifpdf
% {\includegraphics[width=200pt,angle=90]{Figures/system.pdf}}
% \else
% {\includegraphics[width=200pt]{Figures/system.eps}}
% \fi
% \end{frame}
%------------------------------------------------------------
% \begin{frame}{Watched literals}\label{slide:watched:literals}
%   \begin{description}
%   \item[Observation] A nogood becomes unit, if all but one of its literals become true
%   \item[Before] Usage of counters following the Dowling/Gallier Algorithm
%     \pause
%   \item[Idea] \
%     \begin{itemize}
%     \item Assign two non-false \alert{watched literals} per nogood
%     \item Only if one of the watched literals becomes false,
%       the nogood is inspected:
%       \begin{itemize}
%       \item Try to find another non-false watched literal
%       \item If no such literal exists, the nogood has become unit or obsolete
%       \end{itemize}
%     \end{itemize}
%     \pause
%   \item[Advantage] No updating when backjumping!
%   \end{description}
% \end{frame}
%------------------------------------------------------------
% \begin{frame}{Summary}
%   \begin{itemize}
%   \item We provided a \alert{uniform approach to ASP solving}, allowing for a
%     transparent \alert{technology transfer from CP and SAT}
%   \item The idea is to view ASP inferences as \alert{unit propagation on nogoods},
%     reflecting \alert{constraints} from program rules, unfounded sets, and conflicts
%   \item We have provided a \alert{conflict-driven algorithm for ASP} solving, using
%     state-of-the-art SAT solving techniques
%   \item Our approach \alert{favors local propagation} on explicit nogoods over unfounded
%     set checks, which explicate inherent loop nogoods that give rise to unit
%     propagation
%   \item Our empirical results show that our ASP solver \alert{\clasp{} is competitive}
%     with and sometimes superior to existing ASP solvers
%   \end{itemize}
% \end{frame}
%------------------------------------------------------------
%----------------------------------------------------------------
%
%%% Local Variables:
%%% mode: latex
%%% TeX-PDF-mode: t
%%% TeX-master: "asp"
%%% End:
